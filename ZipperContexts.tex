\documentclass{beamer}
\usepackage[utf8]{inputenc}
\usepackage{babel}
\usepackage{xcolor}
\usepackage{hyperref}
\hypersetup{
  colorlinks = true,
  linkcolor = blue,
  urlcolor  = blue,
  citecolor = blue,
  anchorcolor = blue
}
\usepackage{Commands}

\usetheme{Darmstadt}

\title{Zipper Contexts}
\date{}

\begin{document}

\frame{\titlepage}

\section{Zipper Contexts}

\begin{frame}{Intro}

What we do here is explore an interesting idea. In some algorithms for bidirectional typechecking of polymorphic lambda calculus, there appears a special thing called a mark. The idea is that the mark designates the place in the context up until which type variables can be generalized when forming universally quantified types.

\vspace{2em}

Since such a mark is just another way of extending the context, besides $\extend{\Gamma}{x}{A}$ (which corresponds to $\fun{x}[A]{e}$) and $\extendtype{\Gamma}{\alpha}$ which corresponds to $\tfun{\alpha}{e}$, it might be worthwile to explore what would happen if we had more ways of extending the context that correspond to terms?

\vspace{2em}

The answer is interesting: contexts would turn into zippers!

\end{frame}

\newcommand{\ctxmark}{\circ}
\NewDocumentCommand{\extendfun}{m m m o}{#1, \fun{#2}{\ctxmark}, #2 : #3 \IfValueTF{#4}{:= #4}{}}
\newcommand{\extendappl}[2]{#1, \app{\ctxmark}{#2}}
\newcommand{\extendappr}[2]{#1, \app{#2}{\ctxmark}}
\newcommand{\extendpairl}[2]{#1, \pair{\ctxmark}{#2}}
\newcommand{\extendpairr}[2]{#1, \pair{#2}{\ctxmark}}
\newcommand{\extendoutl}[1]{#1, \outl{\ctxmark}}
\newcommand{\extendoutr}[1]{#1, \outr{\ctxmark}}
\newcommand{\extendinl}[1]{#1, \inl{\ctxmark}}
\newcommand{\extendinr}[1]{#1, \inr{\ctxmark}}
\newcommand{\extendcasearg}[3]{#1, \case{\ctxmark}{#2}{#3}}
\NewDocumentCommand{\extendcasel}{m m m m m o}{#1, \case{#2}{#3.\, \ctxmark}{#5}, #3 : #4 \IfValueTF{#6}{ := #6}{}}
\NewDocumentCommand{\extendcaser}{m m m m m o}{#1, \case{#2}{#3}{#4.\, \ctxmark}, #4 : #5 \IfValueTF{#6}{ := #6}{}}
\newcommand{\extendemptyelim}[1]{#1, \emptyelim{\ctxmark}}

\begin{frame}{Contexts as zippers}

Contexts: \\
$\Gamma ::=$ \\
\qquad $\emptytypingctx \pipe$ \\
\qquad $\extendfun{\Gamma}{x}{A} \pipe \extendappl{\Gamma}{e_2} \pipe \extendappr{\Gamma}{e_1} \pipe$ \\
\qquad $\extendpairl{\Gamma}{e_2} \pipe \extendpairr{\Gamma}{e_1} \pipe \extendoutl{\Gamma} \pipe \extendoutr{\Gamma} \pipe$ \\
\qquad $\extendinl{\Gamma} \pipe \extendinr{\Gamma} \pipe \extendcasearg{\Gamma}{f}{g} \pipe$ \\
\qquad $\extendcasel{\Gamma}{e}{x}{A}{g} \pipe \extendcaser{\Gamma}{e}{f}{x}{A} \pipe$ \\
\qquad $\extendemptyelim{\Gamma}$

\end{frame}

\begin{frame}{Contexts}

To be clear, $(x : A) \in \Gamma$ if and only if either $\Gamma = \extendfun{\Gamma_L}{x}{A}, \Gamma_R$, or $\Gamma = \extendfun{\Gamma_L}{x}{A}[e], \Gamma_R$, or $\Gamma = \extendcasel{\Gamma}{e}{x}{A}{y.\, e_2}$

\end{frame}

\begin{frame}{Terms}

Terms: \\
$e ::=$ \\
\qquad $x \pipe$ \\
\qquad $\fun{x}{e} \pipe \app{e_1}{e_2} \pipe$ \\
\qquad $\pair{e_1}{e_2} \pipe \outl{e} \pipe \outr{e} \pipe$ \\
\qquad $\inl{e} \pipe \inr{e} \pipe \case{e}{e_1}{e_2} \pipe$ \\
\qquad $\unit \pipe \emptyelim{e}$

\end{frame}

\begin{frame}{Declarative typing -- basics}

\begin{center}
  $\infrule[Var]{\sidecond{(x : A) \in \Gamma}}{\typing{x}{A}}$
\end{center}

\end{frame}

\begin{frame}{Declarative typing -- type-directed rules}

\begin{center}
  $\infrule{\typing[\extendfun{\Gamma}{x}{A}]{e}{B}}{\typing{\fun{x}{e}}{\Fun{A}{B}}}$ \quad
  $\infrule{\typing[\extendappl{\Gamma}{a}]{f}{\Fun{A}{B}} \quad \typing[\extendappr{\Gamma}{f}]{a}{A}}{\typing{\app{f}{a}}{B}}$

  \vspace{2em}

  $\infrule{\typing[\extendpairl{\Gamma}{b}]{a}{A} \quad \typing[\extendpairr{\Gamma}{a}]{b}{B}}{\typing{\pair{a}{b}}{\Prod{A}{B}}}$

  \vspace{2em}

  $\infrule{\typing[\extendoutl{\Gamma}]{e}{\Prod{A}{B}}}{\typing{\outl{e}}{A}}$ \quad
  $\infrule{\typing[\extendoutr{\Gamma}]{e}{\Prod{A}{B}}}{\typing{\outr{e}}{B}}$

  \vspace{2em}

  $\infrule{\typing[\extendinl{\Gamma}]{e}{A}}{\typing{\inl{e}}{\Sum{A}{B}}}$ \quad
  $\infrule{\typing[\extendinr{\Gamma}]{e}{B}}{\typing{\inr{e}}{\Sum{A}{B}}}$
\end{center}

\end{frame}

\begin{frame}{Declarative typing -- type-directed rules}

\begin{center}
  $\infrule{\begin{array}{c} \typing[\extendcasearg{\Gamma}{x.\, e_1}{y.\, e_2}]{e}{\Sum{A}{B}} \\ \typing[\extendcasel{\Gamma}{e}{x}{A}{y.\, e_2}]{e_1}{C} \\ \typing[\extendcaser{\Gamma}{e}{x.\, e_1}{y}{B}]{e_2}{C} \end{array}}{\typing{\case{e}{x.\, e_1}{y.\, e_2}}{C}}$

  \vspace{2em}

  $\infrule{}{\typing{\unit}{\Unit}}$ \quad
  $\infrule{\typing[\extendemptyelim{\Gamma}]{e}{\Empty}}{\typing{\emptyelim{e}}{A}}$
\end{center}

\end{frame}

\begin{frame}{What's the point?}

What is the point of it? By now I don't really remember, but I think it was that we add definitions to the context, i.e. $\extend{\Gamma}{x}{A}[e]$, and then we turn the term-related context extensions into smart constructors, which can turn ordinary variable bindings into definitions.

\vspace{1em}

$\extendfun{\extendappl{\Gamma}{e}}{x}{A} \leadsto \extendfun{\extendappl{\Gamma}{e}}{x}{A}[e]$

\vspace{1em}

$\extendcasel{\Gamma}{(\inl{e})}{x}{A}{y.\, e_2} \leadsto \extendcasel{\Gamma}{(\inl{e})}{x}{A}{y.\, e_2}[e]$

\vspace{1em}

$\extendcaser{\Gamma}{(\inr{e})}{x.\, e_1}{y}{B} \leadsto \extendcaser{\Gamma}{(\inr{e})}{x.\, e_1}{y}{B}[e]$

\end{frame}

\end{document}