\documentclass{beamer}
\usepackage[utf8]{inputenc}
\usepackage{babel}
\usepackage{xcolor}
\usepackage{hyperref}
\hypersetup{
  colorlinks = true,
  linkcolor = blue,
  urlcolor  = blue,
  citecolor = blue,
  anchorcolor = blue
}
\usepackage{Commands}

\usetheme{Darmstadt}

\title{Hinting\\A nice presentation of algorithmic typing}
\author{Wojciech Kołowski}
\date{}

\begin{document}

\frame{\titlepage}

\section{Intro}

\begin{frame}{STLC with Hints}

STLC with Hints is a flavour of STLC inspired by Bidirectional STLC. The main insight behind it is that in Bidirectional STLC, we have a hard time deciding whether a rule should be in checking mode or in inference mode, so why not both? This way, we would have some input type that guides us, but also produce an output type, which is in some sense ``better''. This is a bit silly if we already have the correct type as input, but we can make this idea work by introducing hints, which are types with holes, and insisting that the input is not a type, but merely a hint.

\end{frame}

\section{Hints}

\begin{frame}{Hints}

$H ::= \Hole \pipe \Fun{H_1}{H_2} \pipe \Prod{H_1}{H_2} \pipe \Sum{H_1}{H_2} \pipe \Unit \pipe \Empty$

\vspace{2em}

Intuitively, hints are partial types. They are built like types, except that there's one additional constructor, $\Hole$, which can be read as ``hole'' or ``unknown''.

\vspace{2em}

We use the letter $H$ for hints. When we use letters like $A, B, C$ which usually stand in for types, it means that the hint \textbf{is} a type, i.e. it doesn't contain any $\Hole$s.

\end{frame}

\begin{frame}{Order on hints}

\begin{center}
  $\infrule{}{\hintorder{\Hole}{H}}$

  \vspace{2em}

  $\infrule{\hintorder{H_1}{H_1'} \quad \hintorder{H_2}{H_2'}}{\hintorder{\Fun{H_1}{H_2}}{\Fun{H'_1}{H'_2}}}$

  \vspace{2em}

  $\infrule{\hintorder{H_1}{H_1'} \quad \hintorder{H_2}{H_2'}}{\hintorder{\Prod{H_1}{H_2}}{\Prod{H_1'}{H_2'}}}$

  \vspace{2em}

  $\infrule{\hintorder{H_1}{H_1'} \quad \hintorder{H_2}{H_2'}}{\hintorder{\Sum{H_1}{H_2}}{\Sum{H_1'}{H_2'}}}$

  \vspace{2em}

  $\infrule{}{\hintorder{\Unit}{\Unit}}$ \quad
  $\infrule{}{\hintorder{\Empty}{\Empty}}$
\end{center}

\end{frame}

\begin{frame}{Order on hints -- intuition}

The order can be intuitively interpreted as information increase: $\hintorder{H_1}{H_2}$ means that hint $H_2$ is more informative than $H_1$, but in a compatible way. In other words, $H_1$ and $H_2$ have the same structure, but some $\Hole$s from $H_1$ were possibly refined to something more informative in $H_2$.

\end{frame}

\begin{frame}{Least upper bound of hints}

\begin{center}
  $\hintlub{\Hole}{H} = H$ \\
  $\hintlub{H}{\Hole} = H$ \\
  $\hintlub{(\Fun{H_1}{H_2})}{(\Fun{H'_1}{H'_2})} = \Fun{(\hintlub{H_1}{H'_1})}{(\hintlub{H_2}{H'_2})}$ \\
  $\hintlub{(\Prod{H_1}{H_2})}{(\Prod{H'_1}{H'_2})} = \Prod{(\hintlub{H_1}{H'_1})}{(\hintlub{H_2}{H'_2})}$ \\
  $\hintlub{(\Sum{H_1}{H_2})}{(\Sum{H'_1}{H'_2})} = \Sum{(\hintlub{H_1}{H'_1})}{(\hintlub{H_2}{H'_2})}$ \\
  $\hintlub{\Unit}{\Unit} = \Unit$ \\
  $\hintlub{\Empty}{\Empty} = \Empty$
\end{center}

The order on hints induces a partial operation $\hintlub{}{}$, which computes the least upper bound of two hints when it exists. Intuitively, $\hintlub{}{}$ combines two hints which share the same structure, filling the $\Hole$s in the leaves with something more informative coming from the other argument. For hints with incompatible structure the result is undefined.

\end{frame}

\begin{frame}{Least upper bound of hints -- properties}

If all relevant results are defined, then:

\begin{itemize}
  \item $\hintlub{(\hintlub{H_1}{H_2})}{H_3} = \hintlub{H_1}{(\hintlub{H_2}{H_3})}$
  \item $\hintlub{H_1}{H_2} = \hintlub{H_2}{H_1}$
  \item $\hintlub{\Hole}{H} = H = \hintlub{H}{\Hole}$
  \item $\hintlub{H}{H} = H$
\end{itemize}

\vspace{2em}

If $\hintlub{}{}$ were not partial, $(H, \hintlub{}{}, \Hole)$ would be a commutative idempotent monoid. But since it is partial, meh...

\end{frame}

\begin{frame}{Greatest lower bound of hints}

\begin{center}
  $\hintglb{\Hole}{H} = \Hole$ \\
  $\hintglb{H}{\Hole} = \Hole$ \\
  $\hintglb{(\Fun{H_1}{H_2})}{(\Fun{H'_1}{H'_2})} = \Fun{(\hintglb{H_1}{H'_1})}{(\hintglb{H_2}{H'_2})}$ \\
  $\hintglb{(\Prod{H_1}{H_2})}{(\Prod{H'_1}{H'_2})} = \Prod{(\hintglb{H_1}{H'_1})}{(\hintglb{H_2}{H'_2})}$ \\
  $\hintglb{(\Sum{H_1}{H_2})}{(\Sum{H'_1}{H'_2})} = \Sum{(\hintglb{H_1}{H'_1})}{(\hintglb{H_2}{H'_2})}$ \\
  $\hintglb{\Unit}{\Unit} = \Unit$ \\
  $\hintglb{\Empty}{\Empty} = \Empty$
\end{center}

The order on hints induces a partial operation $\hintglb{}{}$, which computes the greatest lower bound of two hints when it exists. Intuitively, $\hintglb{}{}$ combines two hints which share the same structure, replacing any subhints with the $\Hole$s if it appears in the other argument. For hints with incompatible structure the result is undefined.

\end{frame}

\begin{frame}{Greatest lower bound of hints -- properties}

If all relevant results are defined, then:

\begin{itemize}
  \item $\hintglb{(\hintglb{H_1}{H_2})}{H_3} = \hintglb{H_1}{(\hintglb{H_2}{H_3})}$
  \item $\hintglb{H_1}{H_2} = \hintglb{H_2}{H_1}$
  \item $\hintglb{\Hole}{H} = \Hole = \hintglb{H}{\Hole}$
  \item $\hintglb{H}{H} = H$
\end{itemize}

\vspace{2em}

If $\hintglb{}{}$ were not partial, $(H, \hintglb{}{}, \Hole)$ would be a commutative idempotent monoid. But since it is partial, meh...

\end{frame}

\begin{frame}{Hint subtraction}

\begin{center}
  $\hintdiff{\Hole}{H} = \Hole$ \\
  $\hintdiff{H}{\Hole} = H$ \\
  $\hintdiff{H}{H} = \Hole$ \\
  $\hintdiff{(\Fun{H_1}{H_2})}{(\Fun{H'_1}{H'_2})} = \Fun{(\hintdiff{H_1}{H'_1})}{(\hintdiff{H_2}{H'_2})}$ \\
  $\hintdiff{(\Prod{H_1}{H_2})}{(\Prod{H'_1}{H'_2})} = \Prod{(\hintdiff{H_1}{H'_1})}{(\hintdiff{H_2}{H'_2})}$ \\
  $\hintdiff{(\Sum{H_1}{H_2})}{(\Sum{H'_1}{H'_2})} = \Sum{(\hintdiff{H_1}{H'_1})}{(\hintdiff{H_2}{H'_2})}$
\end{center}

\vspace{2em}

Hint subtraction is a partial operation which we'll need when proving minimality. When subtracting a hint from itself or from $\Hole$, the result is $\Hole$, and subtracting $\Hole$ changes nothing. In the remaining cases, when the hints are not equal but have the same structure, the subtraction proceeds recursively.

\end{frame}

\begin{frame}{Hint subtraction -- properties}

\begin{itemize}
  \item If $\hintorder{H_1}{H_2}$, then $\hintorder{\hintdiff{H_1}{H}}{\hintdiff{H_2}{H}}$
\end{itemize}

\end{frame}

\begin{frame}{Information order on terms}

The order on hints induces an order on terms: it is the smallest relation which preserves term constructors and subsumes hint ordering on annotated terms.

\vspace{2em}

\begin{center}
  $\infrule{\termorder{e_1}{e_2} \quad \hintorder{H_1}{H_2}}{\termorder{\annot{e_1}{H_1}}{\annot{e_2}{H_2}}}$
\end{center}

\vspace{2em}

Intuitively, $\termorder{e_1}{e_2}$ holds when $e_2$ has more informative hints than $e_1$.

\end{frame}

\begin{frame}{Information order on terms -- rules}

\begin{center}
  $\infrule{}{\termorder{x}{x}}$ \quad
  $\infrule{\termorder{e_1}{e_2}}{\termorder{\fun{x}{e_1}}{\fun{x}{e_2}}}$ \quad
  $\infrule{\termorder{f_1}{f_2} \quad \termorder{a_1}{a_2}}{\termorder{\app{f_1}{a_1}}{\app{f_2}{a_2}}}$

  \vspace{2em}

  $\infrule{\termorder{a_1}{a_2} \quad \termorder{b_1}{b_2}}{\termorder{\pair{a_1}{b_1}}{\pair{a_2}{b_2}}}$ \quad
  $\infrule{\termorder{e_1}{e_2}}{\termorder{\outl{e_1}}{\outl{e_2}}}$ \quad
  $\infrule{\termorder{e_1}{e_2}}{\termorder{\outl{e_1}}{\outr{e_2}}}$

  \vspace{2em}

  $\infrule{\termorder{e_1}{e_2}}{\termorder{\inl{e_1}}{\inl{e_2}}}$ \quad
  $\infrule{\termorder{e_1}{e_2}}{\termorder{\inr{e_1}}{\inr{e_2}}}$

  \vspace{2em}

  $\infrule{\termorder{e_1}{e_2} \quad \termorder{f_1}{f_2} \quad \termorder{g_1}{g_2}}{\termorder{\case{e_1}{f_1}{g_1}}{\case{e_2}{f_2}{g_2}}}$

  \vspace{2em}

  $\infrule{}{\termorder{\unit}{\unit}}$ \quad
  $\infrule{\termorder{e_1}{e_2}}{\termorder{\exfalso{e_1}}{\exfalso{e_2}}}$
\end{center}

\end{frame}

\begin{frame}{Hints for term constructors}

\begin{center}
  $\hintfor{\fun{x}{e}} = \Fun{\Hole}{\Hole}$ \\
  $\hintfor{\pair{e_1}{e_2}} = \Prod{\Hole}{\Hole}$ \\
  $\hintfor{\inl{e}} = \Sum{\Hole}{\Hole}$ \\
  $\hintfor{\inr{e}} = \Sum{\Hole}{\Hole}$ \\
  $\hintfor{\unit} = \Unit$
\end{center}

\end{frame}

\section{Declarative typing}

\begin{frame}{Terms}

Terms: \\
$e ::=$ \\
\qquad $x \pipe \termdiff{\annot{e}{H}} \pipe $ \\
\qquad $\fun{x}{e} \pipe \app{e_1}{e_2} \pipe$ \\
\qquad $\pair{e_1}{e_2} \pipe \outl{e} \pipe \outr{e} \pipe$ \\
\qquad $\inl{e} \pipe \inr{e} \pipe \case{e}{e_1}{e_2} \pipe$ \\
\qquad $\unit \pipe \exfalso{e}$

\vspace{2em}

Note: red color marks differences from Bidirectional STLC.

\vspace{2em}

Judgements: \\
$\fullhinting{\Gamma}{e}{H}{A}$ -- in context $\Gamma$, term $e$ checks with hint $H$ and infers type $A$

\end{frame}

\begin{frame}{Declarative typing -- differences}

\begin{center}
  $\infrule[Annot]{\typing{e}{A} \quad \sidecond{\hintorder{H}{A}}}{\typing{\annot{e}{H}}{A}}$
\end{center}

\end{frame}

\section{Algorithmic typing}

\begin{frame}{Algorithmic typing -- basic rules}

\begin{center}
  $\infrule[Var]{\sidecond{(x : A) \in \Gamma} \quad \sidecond{\hintorder{H}{A}}}{\hinting{x}{H}{A}}$

  \vspace{2em}

  $\infrule[Annot]{\hinting{e}{\hintlub{H_1}{H_2}}{A}}{\hinting{\annot{e}{H_1}}{H_2}{A}}$

  \vspace{2em}

  $\infrule[Hole]{\hinting{e}{\hintfor{e}}{A} \quad \sidecond{e\ \texttt{constructor}}}{\hinting{e}{\Hole}{A}}$
\end{center}

\vspace{2em}

Note that the rule $\rulename{Hole}$ can only be applied once, because $\hintfor{e}$ can never be $\Hole$. After applying $\rulename{Hole}$, the only applicable rules are the type-directed ones.

\end{frame}

\begin{frame}{Algorithmic typing -- type-directed rules}

\begin{center}
  $\infrule{\fullhinting{\extend{\Gamma}{x}{A}}{e}{H}{B}}{\hinting{\fun{x}{e}}{\Fun{A}{H}}{\Fun{A}{B}}}$

  \vspace{2em}

  $\infrule{\hinting{f}{\Fun{\Hole}{H}}{\Fun{A}{B}} \quad \hinting{a}{A}{A}}{\hinting{\app{f}{a}}{H}{B}}$

  \vspace{2em}

  $\infrule{\hinting{a}{H_A}{A} \quad \hinting{b}{H_B}{B}}{\hinting{\pair{a}{b}}{\Prod{H_A}{H_B}}{\Prod{A}{B}}}$

  \vspace{2em}

  $\infrule{\hinting{e}{\Prod{H}{\Hole}}{\Prod{A}{B}}}{\hinting{\outl{e}}{H}{A}}$ \quad
  $\infrule{\hinting{e}{\Prod{\Hole}{H}}{\Prod{A}{B}}}{\hinting{\outr{e}}{H}{B}}$
\end{center}

\end{frame}

\begin{frame}{Algorithmic typing -- type-directed rules}

\begin{center}
  $\infrule{\hinting{e}{H}{A}}{\hinting{\inl{e}}{\Sum{H}{B}}{\Sum{A}{B}}}$

  \vspace{2em}

  $\infrule{\hinting{e}{H}{B}}{\hinting{\inr{e}}{\Sum{A}{H}}{\Sum{A}{B}}}$

  \vspace{2em}

  $\infrule{\hinting{e}{\Sum{\Hole}{\Hole}}{\Sum{A}{B}} \quad \begin{array}{c} \hinting{f}{\Fun{A}{H}}{\Fun{A}{C}} \\ \hinting{g}{\Fun{B}{C}}{\Fun{B}{C}} \end{array}}{\hinting{\case{e}{f}{g}}{H}{C}}$

  \vspace{2em}

  $\infrule{}{\hinting{\unit}{\Unit}{\Unit}}$ \quad
  $\infrule{\hinting{e}{\Empty}{\Empty}}{\hinting{\exfalso{e}}{A}{A}}$
\end{center}

\end{frame}

\begin{frame}{Algorithmic typing -- alternative rules}

\begin{center}
  $\infrule[AltApp]{\hinting{a}{\Hole}{A} \quad \hinting{f}{\Fun{A}{H}}{\Fun{A}{B}}}{\hinting{\app{f}{a}}{H}{B}}$

  \vspace{2em}

  $\infrule[AltCase]{\begin{array}{c} \hinting{f}{\Fun{\Hole}{H}}{\Fun{A}{C}} \\ \hinting{g}{\Fun{\Hole}{C}}{\Fun{B}{C}} \end{array} \quad \hinting{e}{\Sum{A}{B}}{\Sum{A}{B}}}{\hinting{\case{e}{f}{g}}{H}{C}}$
\end{center}

\vspace{2em}

We could have made some different choices. For application, we could try to infer the argument type first and then feed it to the function as a hint. For case, we could try to infer domains of the branches first, then feed these as hints when checking the discriminee.

\end{frame}

\section{Metatheory}

\begin{frame}{Metatheory -- basics}

If $\fullhinting{\Gamma}{e}{H}{A}$ then:

\begin{itemize}
  \item (Soundness) $\fulltyping{\Gamma}{e}{A}$ (proof: induction)
  \item (Compatibility) $\hintorder{H}{A}$ (proof: induction)
  \item (Squeeze) If $\hintorder{H}{H'}$ and $\hintorder{H'}{A}$, then $\fullhinting{\Gamma}{e}{H'}{A}$ (proof: induction)
  \item (Decidability) For $\Gamma, e, H$ it is decidable whether there exists $A$ such that $\fullhinting{\Gamma}{e}{H}{A}$ (proof: the rules are literally the algorithm)
  \item (Determinism) If $\hinting{e}{H}{A}$ and $\hinting{e}{H}{B}$, then $A = B$.
\end{itemize}

\end{frame}

\begin{frame}{Metatheory -- greatest lower bound}

If $\hinting{e}{H_1}{A}$ and $\hinting{e}{H_2}{A}$, then $\hinting{e}{\hintglb{H_1}{H_2}}{A}$ (TODO: not proven, probably not true; fails for $\rulename{App}$, but works for $\rulename{AltApp}$; also fails for $\rulename{Outl}$ and $\rulename{Outr}$)

\end{frame}

\begin{frame}{Metatheory -- minimality}

(Minimality) There exists $\hintorder{H'}{H}$ such that $\fullhinting{\Gamma}{e}{H'}{A}$ and for all $\stricthintorder{H''}{H'}$ it is not the case that $\fullhinting{\Gamma}{e}{H''}{A}$ (proof: induction, the only hard case is $\rulename{Annot}$)

\vspace{1em}

Proof: we need $\hinting{\annot{e}{H_1}}{H'}{A}$ for minimal $H'$. From the induction hypothesis we have minimal $\hintorder{H'}{\hintlub{H_1}{H_2}}$ such that $\hinting{e}{H'}{A}$. Our minimal hint will be $\hintdiff{H'}{H_1}$, so we need $\hinting{\annot{e}{H_1}}{\hintdiff{H'}{H_1}}{A}$. Since $\hintlub{H_1}{(\hintdiff{H'}{H_1})} = \hintlub{H_1}{H'}$, it suffices to show $\hinting{e}{\hintlub{H_1}{H'}}{A}$, which follows from squeezing and IH, because $\triplehintorder{H'}{\hintlub{H_1}{H'}}{\hintlub{H_1}{H_2}}$. Of course we also have $\triplehintorder{\hintdiff{H'}{H_1}}{\hintdiff{H_2}{H_1}}{H_2}$.

\end{frame}

\begin{frame}{Metatheory -- minimality, cont.}

Now assume $\stricthintorder{H''}{\hintdiff{H'}{H_1}}$ and $\hinting{\annot{e}{H_1}}{H''}{A}$. We have $\stricthintorder{H''}{H'}$ and so $\stricthintorder{H''}{H_2}$.

\vspace{2em}

TODO

\end{frame}

\begin{frame}{Metatheory -- minimality v2}

(Minimality v2) There exists $\hintorder{M}{H}$ such that $\fullhinting{\Gamma}{e}{M}{A}$ and for all $\hintorder{M'}{M}$ such that $\fullhinting{\Gamma}{e}{M'}{A}$ we have $M' = M$ (proof: TODO)

\vspace{1em}

\end{frame}

\begin{frame}{Metatheory 2}

If $\fulltyping{\Gamma}{e}{A}$, then:

\begin{itemize}
  \item (Annotability, checking) There exists $e'$ such that $\termorder{e}{e'}$ and $\fullhinting{\Gamma}{e'}{A}{A}$ (proof: induction; annotations need to be put on eliminators, like in Dual Intrinsic STLC)
  \item (Annotability, inference) There exists $e'$ such that $\termorder{e}{e'}$ and $\fullhinting{\Gamma}{e'}{\Hole}{A}$ (proof: induction; annotations need to be put on constructors, like in Intrinsic STLC)
  \item (Minimal annotability) There exists $e'$ such that $\termorder{e}{e'}$ and $\fullhinting{\Gamma}{e'}{A}{A}$ and for all $e''$ such that $\termorder{e}{e''}$ and $\termorder{e''}{e'}$ it is not the case that $\fullhinting{\Gamma}{e''}{A}{A}$
  \item There exist $\termorder{e}{e_1}$ and $\hintorder{H_1}{A}$ such that $\fullhinting{\Gamma}{e_1}{H_1}{A}$ and for all $\tripletermorder{e}{e_2}{e_1}$ and $\triplehintorder{H}{H'}{A}$ it is not the case that $\fullhinting{\Gamma}{e_2}{H_2}{A}$
\end{itemize}

\end{frame}

\begin{frame}{(Non)uniqueness of typing}

Similarly to Extrinsic STLC, STLC with Hints does not enjoy uniqueness of typing. This is because we still can have terms like $\fun{x}{x}$ with hint $\Hole$, which can be typed with any type of the form $\Fun{A}{A}$. However, if the hint is informative enough, then the type is unique. Moreover, every typable term can be given a hint which makes its type unique.

\end{frame}

\section{Embedding}

\begin{frame}{Embedding Bidirectional STLC}

We will now turn our attention towards showing that Bidirectional STLC can be in some sense ``embedded'' into our hint-based system.

\vspace{2em}

First, checking and inference modes are definable:

\begin{itemize}
  \item $\check{e}{A}$ is defined as $\hinting{e}{A}{A}$
  \item $\infer{e}{A}$ is defined as $\hinting{e}{\Hole}{A}$
\end{itemize}

\vspace{2em}

Second, we will show that bidirectional typing rules are admissible in the hint-based system.

\end{frame}

\begin{frame}{Embedding Bidirectional STLC -- variables and annotations}

\begin{center}
  $\infrule[Var]{\sidecond{(x : A) \in \Gamma}}{\infer{x}{A}}$

  \vspace{2em}

  $\infrule[Annot]{\check{e}{A}}{\infer{\annot{e}{A}}{A}}$
\end{center}

\vspace{2em}

Thanks to our definitions, the bidirectional rules $\rulename{Var}$ and $\rulename{Annot}$ are special cases of our hint-based $\rulename{Var}$ and $\rulename{Annot}$.

\end{frame}

\begin{frame}{Embedding Bidirectional STLC -- subsumption}

\begin{center}
  $\infrule[Sub]{\infer{e}{B} \quad \sidecond{A = B}}{\check{e}{A}}$
\end{center}

\vspace{2em}

Subsumption is a bit harder to prove, but it boils down to our squeeze theorem. After rewriting $A = B$, we have $\hinting{e}{\Hole}{A}$ and we need to show $\hinting{e}{A}{A}$, but this follows from squeeze theorem because $\triplehintorder{\Hole}{A}{A}$.

\end{frame}

\begin{frame}{Embedding Bidirectional STLC -- checked rules}

\begin{center}
  $\infrule{\fullcheck{\extend{\Gamma}{x}{A}}{e}{B}}{\check{\fun{x}{e}}{\Fun{A}{B}}}$

  \vspace{2em}

  $\infrule{\check{a}{A} \quad \check{b}{B}}{\check{\pair{a}{b}}{\Prod{A}{B}}}$

  \vspace{2em}

  $\infrule{\check{e}{A}}{\check{\inl{e}}{\Sum{A}{B}}}$ \quad
  $\infrule{\check{e}{B}}{\check{\inr{e}}{\Sum{A}{B}}}$

  \vspace{2em}

  $\infrule{}{\check{\unit}{\Unit}}$ \quad
  $\infrule{\check{e}{\Empty}}{\check{\exfalso{e}}{A}}$
\end{center}

\vspace{1em}

Checked rules are special cases of the corresponding hint-based rules.

\end{frame}

\begin{frame}{Embedding Bidirectional STLC -- rules with inference}

\begin{center}
  $\infrule{\infer{f}{\Fun{A}{B}} \quad \check{a}{A}}{\infer{\app{f}{a}}{B}}$

  \vspace{2em}

  $\infrule{\infer{e}{\Prod{A}{B}}}{\infer{\outl{e}}{A}}$ \quad
  $\infrule{\infer{e}{\Prod{A}{B}}}{\infer{\outr{e}}{B}}$

  \vspace{2em}

  $\infrule{\infer{e}{\Sum{A}{B}} \quad \check{f}{\Fun{A}{C}} \quad \check{g}{\Fun{B}{C}}}{\check{\case{e}{f}{g}}{C}}$
\end{center}

\vspace{1em}

To prove rules which make use of inference mode, we need to use the squeeze theorem. For example, assume $\hinting{f}{\Hole}{\Fun{A}{B}}$ and $\hinting{a}{A}{A}$. From the squeeze theorem, we also have $\hinting{f}{\Fun{\Hole}{\Hole}}{\Fun{A}{B}}$, so that $\hinting{\app{f}{a}}{\Hole}{B}$ follows from hint-based rule $\rulename{App}$.

\end{frame}

\begin{frame}{Embedding Bidirectional STLC -- additional rules}

\begin{center}
  $\infrule{\infer{e}{\Sum{A}{B}} \quad \infer{f}{\Fun{A}{C}} \quad \infer{g}{\Fun{B}{C}}}{\infer{\case{e}{f}{g}}{C}}$

  \vspace{2em}

  $\infrule{}{\infer{\unit}{\Unit}}$

  \vspace{2em}

  $\infrule{\infer{a}{A} \quad \infer{b}{B}}{\infer{\pair{a}{b}}{\Prod{A}{B}}}$
\end{center}

\vspace{1em}

The inferred rule for case also requires using the squeeze theorem, but this time we need to do it three times, once for each premise. As for the inferred rules for unit and pairs, they follow from the corresponding hint-based rules, but first we need to use the rule $\rulename{Hole}$ to adjust the hint.

\end{frame}

\begin{frame}{Embedding Intrinsic STLC -- terms}

We can also embed Intrinsic STLC. First, we define the missing terms using partial annotations:

\begin{itemize}
  \item $\ifun{x}{A}{e} :\equiv \annot{\fun{x}{e}}{\Fun{A}{\Hole}}$ \\
  \item $\iinl{B}{e} :\equiv \annot{\inl{e}}{\Sum{\Hole}{B}}$ \\
  \item $\iinr{A}{e} :\equiv \annot{\inr{e}}{\Sum{A}{\Hole}}$ \\
  \item $\iexfalso{A}{e} :\equiv \annot{\exfalso{e}}{A}$
\end{itemize}

\vspace{2em}

Second, we will show that the typing rules are admissible in the hint-based system.

\end{frame}

\begin{frame}{Embedding Intrinsic STLC -- rules}

\begin{center}
  $\infrule{\fullinfer{\extend{\Gamma}{x}{A}}{e}{B}}{\infer{\ifun{x}{A}{e}}{\Fun{A}{B}}}$

  \vspace{2em}

  $\infrule{\infer{e}{A}}{\infer{\iinl{B}{e}}{\Sum{A}{B}}}$ \quad
  $\infrule{\infer{e}{B}}{\infer{\iinr{A}{e}}{\Sum{A}{B}}}$

  \vspace{2em}

  $\infrule{\infer{e}{\Empty}}{\infer{\iexfalso{A}{e}}{A}}$
\end{center}

\vspace{2em}

Rules for lambdas and sum constructors can be derived by first using the rule for annotations and then the corresponding hint-based rule. For $\exfalso{}$ we also need to use the squeeze theorem on the premise.

\end{frame}

\begin{frame}{Embedding Dual Intrinsic STLC -- terms}

We can also embed Dual Intrinsic STLC. First, we define the missing terms using partial annotations:

\begin{itemize}
  \item $\iapp{A}{f}{a} :\equiv \app{\annot{f}{\Fun{A}{\Hole}}}{a}$
  \item $\ioutl{B}{e} :\equiv \outl{\annot{e}{\Prod{\Hole}{B}}}$
  \item $\ioutr{A}{e} :\equiv \outr{\annot{e}{\Prod{A}{\Hole}}}$
  \item $\icase{A}{B}{e}{f}{g} :\equiv \case{\annot{e}{\Sum{A}{B}}}{f}{g}$
\end{itemize}

\vspace{2em}

Second, we will show that the typing rules are admissible in the hint-based system.

\end{frame}

\begin{frame}{Embedding Dual Intrinsic STLC -- rules}

\begin{center}
  $\infrule{\check{f}{\Fun{A}{B}} \quad \check{a}{A}}{\check{\iapp{A}{f}{a}}{B}}$

  \vspace{2em}

  $\infrule{\check{e}{\Prod{A}{B}}}{\check{\ioutl{B}{e}}{A}}$ \enspace
  $\infrule{\check{e}{\Prod{A}{B}}}{\check{\ioutr{A}{e}}{B}}$

  \vspace{2em}

  $\infrule{\check{e}{\Sum{A}{B}} \quad \check{f}{\Fun{A}{C}} \quad \check{g}{\Fun{B}{C}}}{\check{\icase{A}{B}{e}{f}{g}}{C}}$
\end{center}

\vspace{2em}

To prove the rules, we first need to use the corresponding hint-based rule and then the annotation rule in one of the premises.

\end{frame}

\end{document}