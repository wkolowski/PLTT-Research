\documentclass{beamer}
\usepackage[utf8]{inputenc}
\usepackage{babel}
\usepackage{xcolor}
\usetheme{Darmstadt}

\newcommand{\pipe}{\ |\ }

\newcommand{\Fun}[2]{#1 \to #2}
\newcommand{\Prod}[2]{#1 \times #2}
\newcommand{\Sum}[2]{#1 + #2}
\newcommand{\Unit}{\textbf{1}}
\newcommand{\Empty}{\textbf{0}}

\newcommand{\annot}[2]{(#1 : #2)}
\newcommand{\fun}[2]{\lambda #1. #2}
\newcommand{\app}[2]{#1\ #2}
\newcommand{\pair}[2]{(#1, #2)}
\newcommand{\outl}[1][]{\texttt{outl}\ #1}
\newcommand{\outr}[1][]{\texttt{outr}\ #1}
\newcommand{\inl}[1][]{\texttt{inl}\ #1}
\newcommand{\inr}[1][]{\texttt{inr}\ #1}
\newcommand{\case}[3]{\texttt{case}\ #1\ \texttt{of}\ (#2, #3)}
\newcommand{\unit}{\texttt{unit}}
\newcommand{\exfalso}[1][]{\texttt{exfalso}\ #1}

\newcommand{\subst}[3]{#1\left[#2 := #3\right]}

\newcommand{\tyctx}[1]{\mathcolor{brown}{#1\ \texttt{ctx}}}

\newcommand{\fulltyping}[3]{#1 \vdash #2 : #3}
\newcommand{\typing}[2]{\fulltyping{\Gamma}{#1}{#2}}

\newcommand{\rulename}[1]{\textsc{\footnotesize{#1}}}
\newcommand{\infrule}[3][]{\displaystyle \frac{#2}{#3} \rulename{#1}}

\newcommand{\emptytypingctx}{\cdot}
\newcommand{\extend}[3]{#1, #2 : #3}
\newcommand{\extenddef}[4]{#1, #2 : #3 := #4}

\newcommand{\sidecond}[1]{#1}

\title{A zoo of STLCs}

\begin{document}

\frame{\titlepage}

\section{Commons}

\begin{frame}{Types, contexts and terms}

Types: \\
$A, B ::= \Fun{A}{B} \pipe \Prod{A}{B} \pipe \Sum{A}{B} \pipe \Unit \pipe \Empty$

\vspace{2em}

Typing contexts: \\
$\Gamma ::= \emptytypingctx \pipe \extend{\Gamma}{x}{A}$

\end{frame}

\section{Extrinsic}

\begin{frame}{Extrinsic STLC}

Extrinsic STLC is the simplest version of typed lambda calculus. The terms are the same as in untyped lambda calculus (with the addition of terms for products, sums, unit and empty). The typing relation takes the form of a type assignment system -- we describe which terms have which types, without worrying about issues such as implementation.

\end{frame}

\begin{frame}{Terms}

Terms: \\
$e ::=$ \\
\qquad $x \pipe$ \\
\qquad $\fun{x}{e} \pipe \app{e_1}{e_2} \pipe$ \\
\qquad $\pair{e_1}{e_2} \pipe \outl[e] \pipe \outr[e] \pipe$ \\
\qquad $\inl[e] \pipe \inr[e] \pipe \case{e}{e_1}{e_2} \pipe$ \\
\qquad $\unit \pipe \exfalso{e}$

\end{frame}

\begin{frame}{Declarative typing -- basics}

\begin{center}
  $\infrule[Var]{\sidecond{(x : A) \in \Gamma}}{\typing{x}{A}}$
\end{center}

\end{frame}

\begin{frame}{Declarative typing -- type-directed rules}

\begin{center}
  $\infrule{\fulltyping{\extend{\Gamma}{x}{A}}{e}{B}}{\typing{\fun{x}{e}}{\Fun{A}{B}}}$ \quad
  $\infrule{\typing{f}{\Fun{A}{B}} \quad \typing{a}{A}}{\typing{\app{f}{a}}{B}}$

  \vspace{2em}

  $\infrule{\typing{a}{A} \quad \typing{b}{B}}{\typing{\pair{a}{b}}{\Prod{A}{B}}}$ \quad
  $\infrule{\typing{e}{\Prod{A}{B}}}{\typing{\outl[e]}{A}}$ \quad
  $\infrule{\typing{e}{\Prod{A}{B}}}{\typing{\outr[e]}{B}}$

  \vspace{2em}

  $\infrule{\typing{e}{A}}{\typing{\inl[e]}{\Sum{A}{B}}}$ \quad
  $\infrule{\typing{e}{B}}{\typing{\inr[e]}{\Sum{A}{B}}}$

  \vspace{2em}

  $\infrule{\typing{e}{\Sum{A}{B}} \quad \typing{f}{\Fun{A}{C}} \quad \typing{g}{\Fun{B}{C}}}{\typing{\case{e}{f}{g}}{C}}$

  \vspace{2em}

  $\infrule{}{\typing{\unit}{\Unit}}$ \quad
  $\infrule{\typing{e}{\Empty}}{\typing{\exfalso[e]}{A}}$
\end{center}

\end{frame}

\begin{frame}{Metatheory}

Extrinsic STLC enjoys strong metatheoretical properties:

\begin{itemize}
  \item Confluence: if a term can be reduced to two different terms, these two can in turn be reduced to a common result.
  \item Termination: computation on well-typed terms always terminates.
  \item Type preservation: computing with a well-typed term preserves its type.
  \item Canonicity: in the empty context, normal forms are inductively generated from term constructors.
\end{itemize}

Intuitively: given a well-typed term, in finite time it computes to another term of the same type which cannot compute anymore. In the empty context, we know the result of this computation must be a constructor.

\end{frame}

\begin{frame}{Uniqueness of typing}

However, extrinsic STLC does not enjoy another important property: \textbf{uniqueness of typing}. This means that there are terms which can be assigned multiple types. For example, $\fun{x}{x}$ can be assigned the type $\Fun{A}{A}$ for any type $A$. There four culprits of this failure are lambda abstractions, sum constructors and exfalso.

\end{frame}

\section{Bidirectional}

\newcommand{\newterm}[1]{\mathcolor{green}{#1}}

\begin{frame}{Bidirectional STLC}

Bidirectional STLC is a version of simply typed lambda calculus which focuses on an efficient implementation of the typechecker. The terms are as in Extrinsic STLC, but with the addition of a general type annotation construct that can appear anywhere and is not mandatory. The typing relation can be presented as type assignment system for reference, but more important are the type checking and type inference relations, both of which are algorithmic, i.e. easily implementable.

\end{frame}

\begin{frame}{Terms}

Note: green color marks terms which were not present in Extrinsic STLC.

\vspace{2em}

Terms: \\
$e ::=$ \\
\qquad $x \pipe \newterm{\annot{e}{A}} \pipe$ \\
\qquad $\fun{x}{e} \pipe \app{e_1}{e_2} \pipe$ \\
\qquad $\pair{e_1}{e_2} \pipe \outl[e] \pipe \outr[e] \pipe$ \\
\qquad $\inl[e] \pipe \inr[e] \pipe \case{e}{e_1}{e_2} \pipe$ \\
\qquad $\unit \pipe \exfalso{e}$

\end{frame}

\begin{frame}{Declarative typing -- new rules}

\begin{center}
  $\infrule[Annot]{\typing{e}{A}}{\typing{\annot{e}{A}}{A}}$
\end{center}

\end{frame}

\newcommand{\fullcheck}[3]{#1 \vdash #2 \mathcolor{blue}{\Leftarrow} #3}
\renewcommand{\check}[2]{\fullcheck{\Gamma}{#1}{#2}}

\newcommand{\fullinfer}[3]{#1 \vdash #2 \mathcolor{red}{\Rightarrow} #3}
\newcommand{\infer}[2]{\fullinfer{\Gamma}{#1}{#2}}

\begin{frame}{Bidirectional typing -- basics}

\begin{center}
  $\infrule[Var]{\sidecond{(x : A) \in \Gamma}}{\infer{x}{A}}$

  \vspace{2em}

  $\infrule[Annot]{\check{e}{A}}{\infer{\annot{e}{A}}{A}}$

  \vspace{2em}

  $\infrule[Sub]{\infer{e}{B} \quad \sidecond{A = B}}{\check{e}{A}}$
\end{center}

\end{frame}

\begin{frame}{Bidirectional typing -- type-directed rules}

\begin{center}
  $\infrule{\fullcheck{\extend{\Gamma}{x}{A}}{e}{B}}{\check{\fun{x}{e}}{\Fun{A}{B}}}$ \quad
  $\infrule{\infer{f}{\Fun{A}{B}} \quad \check{a}{A}}{\infer{\app{f}{a}}{B}}$

  \vspace{2em}

  $\infrule{\check{a}{A} \quad \check{b}{B}}{\check{\pair{a}{b}}{\Prod{A}{B}}}$ \quad
  $\infrule{\infer{e}{\Prod{A}{B}}}{\infer{\outl[e]}{A}}$ \quad
  $\infrule{\infer{e}{\Prod{A}{B}}}{\infer{\outr[e]}{B}}$

  \vspace{2em}

  $\infrule{\check{e}{A}}{\check{\inl[e]}{\Sum{A}{B}}}$ \quad
  $\infrule{\check{e}{B}}{\check{\inr[e]}{\Sum{A}{B}}}$

  \vspace{2em}

  $\infrule{\infer{e}{\Sum{A}{B}} \quad \check{f}{\Fun{A}{C}} \quad \check{g}{\Fun{B}{C}}}{\check{\case{e}{f}{g}}{C}}$

  \vspace{2em}

  $\infrule{}{\check{\unit}{\Unit}}$ \quad
  $\infrule{\check{e}{\Empty}}{\check{\exfalso[e]}{A}}$
\end{center}

\end{frame}

\begin{frame}{Bidirectional typing -- additional rules}

\begin{center}
  $\infrule{}{\infer{\unit}{\Unit}}$

  \vspace{1em}

  $\infrule{\infer{e}{\Sum{A}{B}} \quad \infer{f}{\Fun{A}{C}} \quad \infer{g}{\Fun{B}{C}}}{\infer{\case{e}{f}{g}}{C}}$

  \vspace{1em}

  $\infrule{\infer{a}{A} \quad \infer{b}{B}}{\infer{\pair{a}{b}}{\Prod{A}{B}}}$
\end{center}

\vspace{1em}

The basic rules are as presented in the previous slide. However, it is possible to add some enhancements. First, we can replace the rule for $\unit$ with an inference rulem and if we need to check, we can use subsumption. Second, the paper argues that sum elimination is a general principle, and so it should allow both checking and inference versions. We could also add an inference rule for pairs. It seems there isn't a trade-off either -- if the checking rule fails, we can use subsumption and try to infer.

\end{frame}

\begin{frame}{Metatheory}

Similarly to Extrinsic STLC, typing is not unique in Bidirectional STLC. This is because while we do have annotations in terms, we are not forced to use them. Therefore, we can check terms like $\fun{x}{x}$ with any type of the form $\Fun{A}{A}$. However, inference is unique.

\end{frame}

\section{Intrinsic}

\newcommand{\ifun}[3]{\lambda #1 : #2. #3}
\newcommand{\iinl}[2]{\texttt{inl}_{#1}\ #2}
\newcommand{\iinr}[2]{\texttt{inr}_{#1}\ #2}
\newcommand{\iexfalso}[2]{\texttt{exfalso}_{#1}\ #2}

\newcommand{\termdiff}[1]{\mathcolor{red}{#1}}

\begin{frame}{Intrinsic STLC}

Intrinsic STLC is the most widespread version of typed lambda calculus. The terms differ from Extrinsic STLC, as type annotations are mandatory on lambdas, sum constructors and the empty type eliminator.

\end{frame}

\begin{frame}{Terms}

Note: red color marks places which differ from Extrinsic STLC.

\vspace{2em}

Terms: \\
$e ::=$ \\
\qquad $x \pipe$ \\
\qquad $\termdiff{\ifun{x}{A}{e}} \pipe \app{e_1}{e_2} \pipe$ \\
\qquad $\pair{e_1}{e_2} \pipe \outl[e] \pipe \outr[e] \pipe$ \\
\qquad $\termdiff{\iinl{A}{e}} \pipe \termdiff{\iinr{A}{e}} \pipe \case{e}{e_1}{e_2} \pipe$ \\
\qquad $\unit \pipe \termdiff{\iexfalso{A}{e}}$

\vspace{2em}

\end{frame}

\begin{frame}{Declarative typing -- differences}

\begin{center}
  $\infrule{\fulltyping{\extend{\Gamma}{x}{A}}{e}{B}}{\typing{\ifun{x}{A}{e}}{\Fun{A}{B}}}$

  \vspace{2em}

  $\infrule{\typing{e}{A}}{\typing{\iinl{B}{e}}{\Sum{A}{B}}}$ \quad
  $\infrule{\typing{e}{B}}{\typing{\iinr{A}{e}}{\Sum{A}{B}}}$

  \vspace{2em}

  $\infrule{\typing{e}{\Empty}}{\typing{\iexfalso{A}{e}}{A}}$
\end{center}

\end{frame}

\begin{frame}{Algorithmic typing -- basics}

\begin{center}
  $\infrule[Var]{\sidecond{(x : A) \in \Gamma}}{\infer{x}{A}}$
\end{center}

\end{frame}

\begin{frame}{Algorithmic typing -- type-directed rules}

\begin{center}
  $\infrule{\fullinfer{\extend{\Gamma}{x}{A}}{e}{B}}{\infer{\ifun{x}{A}{e}}{\Fun{A}{B}}}$ \quad
  $\infrule{\infer{f}{\Fun{A}{B}} \quad \infer{a}{A}}{\infer{\app{f}{a}}{B}}$

  \vspace{2em}

  $\infrule{\infer{a}{A} \quad \infer{b}{B}}{\infer{\pair{a}{b}}{\Prod{A}{B}}}$ \quad
  $\infrule{\infer{e}{\Prod{A}{B}}}{\infer{\outl[e]}{A}}$ \quad
  $\infrule{\infer{e}{\Prod{A}{B}}}{\infer{\outr[e]}{B}}$

  \vspace{2em}

  $\infrule{\infer{e}{A}}{\infer{\iinl{B}{e}}{\Sum{A}{B}}}$ \quad
  $\infrule{\infer{e}{B}}{\infer{\iinr{A}{e}}{\Sum{A}{B}}}$

  \vspace{2em}

  $\infrule{\infer{e}{\Sum{A}{B}} \quad \infer{f}{\Fun{A}{C}} \quad \infer{g}{\Fun{B}{C}}}{\infer{\case{e}{f}{g}}{C}}$

  \vspace{2em}

  $\infrule{}{\infer{\unit}{\Unit}}$ \quad
  $\infrule{\infer{e}{\Empty}}{\infer{\iexfalso{A}{e}}{A}}$
\end{center}

\end{frame}

\begin{frame}{Metatheory}

Because of the annotations on lambda, sum constructors and exfalso, Intrinsic STLC does enjoy uniqueness of typing. It is easy to prove this by induction: types of most terms are determined by the induction hypothesis, whereas for the aforementioned four we need to supplement the induction hypothesis with the annotation.

\end{frame}

\section{Dual Intrinsic}

\newcommand{\iapp}[3]{\texttt{app}_{#1}\ #2\ #3}
\newcommand{\ioutl}[2]{\texttt{outl}_{#1}\ #2}
\newcommand{\ioutr}[2]{\texttt{outr}_{#1}\ #2}
\newcommand{\icase}[5]{\texttt{case}_{#1, #2}\ #3\ \texttt{of}\ (#4, #5)}

\begin{frame}{Dual Intrinsic STLC}

If the usual Intrinsic STLC turns out to be a system in which we can infer all types, what would a system in which we can check all types look like?

\end{frame}

\begin{frame}{Terms}

Note: red color marks places which differ from Extrinsic STLC.

\vspace{2em}

Terms: \\
$e ::=$ \\
\qquad $x \pipe$ \\
\qquad $\fun{x}{e} \pipe \termdiff{\iapp{A}{e_1}{e_2}} \pipe$ \\
\qquad $\pair{e_1}{e_2} \pipe \termdiff{\ioutl{A}{e}} \pipe \termdiff{\ioutr{A}{e}} \pipe$ \\
\qquad $\inl[e] \pipe \inr[e] \pipe \termdiff{\icase{A}{B}{e}{e_1}{e_2}} \pipe$ \\
\qquad $\unit \pipe \exfalso{e}$

\end{frame}

\begin{frame}{Declarative typing -- differences}

\begin{center}
  $\infrule{\typing{f}{\Fun{A}{B}} \quad \typing{a}{A}}{\typing{\iapp{A}{f}{a}}{B}}$

  \vspace{2em}

  $\infrule{\typing{e}{\Prod{A}{B}}}{\typing{\ioutl{B}{e}}{A}}$ \quad
  $\infrule{\typing{e}{\Prod{A}{B}}}{\typing{\ioutr{A}{e}}{B}}$ \quad

  \vspace{2em}

  $\infrule{\typing{e}{\Sum{A}{B}} \quad \typing{f}{\Fun{A}{C}} \quad \typing{g}{\Fun{B}{C}}}{\typing{\icase{A}{B}{e}{f}{g}}{C}}$
\end{center}

\end{frame}

\begin{frame}{Algorithmic typing -- basics}

\begin{center}
  $\infrule[Var]{\sidecond{(x : A) \in \Gamma}}{\check{x}{A}}$
\end{center}

\end{frame}

\begin{frame}{Algorithmic typing -- type-directed rules}

\begin{center}
  $\infrule{\fullcheck{\extend{\Gamma}{x}{A}}{e}{B}}{\check{\fun{x}{e}}{\Fun{A}{B}}}$ \enspace
  $\infrule{\check{f}{\Fun{A}{B}} \quad \check{a}{A}}{\check{\iapp{A}{f}{a}}{B}}$

  \vspace{2em}

  $\infrule{\check{a}{A} \quad \check{b}{B}}{\check{\pair{a}{b}}{\Prod{A}{B}}}$ \enspace
  $\infrule{\check{e}{\Prod{A}{B}}}{\check{\ioutl{B}{e}}{A}}$ \enspace
  $\infrule{\check{e}{\Prod{A}{B}}}{\check{\ioutr{A}{e}}{B}}$

  \vspace{2em}

  $\infrule{\check{e}{A}}{\check{\inl[e]}{\Sum{A}{B}}}$ \quad
  $\infrule{\check{e}{B}}{\check{\inr[e]}{\Sum{A}{B}}}$

  \vspace{2em}

  $\infrule{\check{e}{\Sum{A}{B}} \quad \check{f}{\Fun{A}{C}} \quad \check{g}{\Fun{B}{C}}}{\check{\icase{A}{B}{e}{f}{g}}{C}}$

  \vspace{2em}

  $\infrule{}{\check{\unit}{\Unit}}$ \quad
  $\infrule{\check{e}{\Empty}}{\check{\exfalso[e]}{A}}$
\end{center}

\end{frame}

\begin{frame}{Metatheory}

Even though Dual Intrinsic STLC has plenty of mandatory annotations, it does not enjoy uniqueness of typing for the usual reasons: we can type $\fun{x}{x}$ with any type of the form $\Fun{A}{A}$. The role of the annotations is not to force types to be unique, but to make it possible to implement type checking.

\end{frame}

\section{Hints}

\newcommand{\Hint}{\mathbf{?}}

\newcommand{\combinehints}[2]{#1 \sqcap #2}

\newcommand{\hintorder}[2]{#1 \sqsubseteq #2}

\begin{frame}{Hints}

We introduce hints, which will be useful to describe two more variants of STLC.

\vspace{2em}

$H ::= \Hint \pipe \Fun{H_1}{H_2} \pipe \Prod{H_1}{H_2} \pipe \Sum{H_1}{H_2} \pipe \Unit \pipe \Empty$

\vspace{2em}

Intuitively, hints are partial types. They are built like types, except that there's one additional way to make hints: $\Hint$, which can be read as ``unknown'', ``hole'' or ``whatever''.

\vspace{2em}

We use the letter $H$ for hints. When we use letters like $A, B, C$ which usually stand in for types, it means that the hint \textbf{is} a type, i.e. it doesn't contain any $\Hint$s.

\end{frame}

\begin{frame}{Combining hints}

\begin{center}
  $\combinehints{\Hint}{H} = H$ \\
  $\combinehints{H}{\Hint} = H$ \\
  $\combinehints{(\Fun{H_1}{H_2})}{(\Fun{H'_1}{H'_2})} = \Fun{(\combinehints{H_1}{H'_1})}{(\combinehints{H_2}{H'_2})}$ \\
  $\combinehints{(\Prod{H_1}{H_2})}{(\Prod{H'_1}{H'_2})} = \Prod{(\combinehints{H_1}{H'_1})}{(\combinehints{H_2}{H'_2})}$ \\
  $\combinehints{(\Sum{H_1}{H_2})}{(\Sum{H'_1}{H'_2})} = \Sum{(\combinehints{H_1}{H'_1})}{(\combinehints{H_2}{H'_2})}$ \\
  $\combinehints{\Unit}{\Unit} = \Unit$ \\
  $\combinehints{\Empty}{\Empty} = \Empty$
\end{center}

Intuitively, $\combinehints{}{}$ is a partial binary operation which combines two hints which share the same structure, possibly filling the $\Hint$s in the leaves with something more informative coming from the other argument. For hints which have incompatible structure (e.g. function hint and product hint) the result is undefined.

\end{frame}

\begin{frame}{Combining hints -- properties}

If all possible results of $\combinehints{}{}$ are defined, then:

\begin{itemize}
  \item $\combinehints{(\combinehints{H_1}{H_2})}{H_3} = \combinehints{H_1}{(\combinehints{H_2}{H_3})}$
  \item $\combinehints{H_1}{H_2} = \combinehints{H_2}{H_1}$
  \item $\combinehints{\Hint}{H} = H = \combinehints{H}{\Hint}$
  \item $\combinehints{H}{H} = H$.
\end{itemize}

\vspace{2em}

If $\combinehints{}{}$ were not partial, it would be a commutative idempotent monoid. But since it is partial, meh...

\end{frame}

\begin{frame}{Order on hints}

The operation $\combinehints{}{}$ induces an order on hints in the usual way: $\hintorder{H_1}{H_2} :\equiv (\combinehints{H_1}{H_2} = H_2)$.

\vspace{2em}

This order can be intuitively interpreted as an information increase: $\hintorder{H_1}{H_2}$ means that hint $H_2$ is more informative than $H_1$, but in a compatible way. In other words, $H_1$ and $H_2$ have the same structure, but some $\Hint$s from $H_1$ can be refined to something more informative in $H_2$.

\end{frame}

\begin{frame}{Order on hints -- explicitly}


\begin{center}
  $\infrule{}{\hintorder{\Hint}{H}}$

  \vspace{2em}

  $\infrule{\hintorder{H_1}{H_1'} \quad \hintorder{H_2}{H_2'}}{\hintorder{\Fun{H_1}{H_2}}{\Fun{H'_1}{H'_2}}}$

  \vspace{2em}

  $\infrule{\hintorder{H_1}{H_1'} \quad \hintorder{H_2}{H_2'}}{\hintorder{\Prod{H_1}{H_2}}{\Prod{H_1'}{H_2'}}}$

  \vspace{2em}

  $\infrule{\hintorder{H_1}{H_1'} \quad \hintorder{H_2}{H_2'}}{\hintorder{\Sum{H_1}{H_2}}{\Sum{H_1'}{H_2'}}}$

  \vspace{2em}

  $\infrule{}{\hintorder{\Unit}{\Unit}}$ \quad
  $\infrule{}{\hintorder{\Empty}{\Empty}}$
\end{center}

\end{frame}

\section{Hinting v1}

\newcommand{\fullhinting}[4]{#1 \vdash #2 \mathcolor{blue}{\Leftarrow} #3 \mathcolor{red}{\Rightarrow} #4}
\newcommand{\hinting}[3]{\fullhinting{\Gamma}{#1}{#2}{#3}}

\begin{frame}{Hinting v1}

STLC with hints (version 1) is a flavour of STLC inspired by Bidirectional STLC. The main insight behind it is that in Bidirectional STLC, we have a hard time deciding whether a rule should be in checking mode or in inference mode, so why not both? This way, we would have some input type that guides us, but also produce an output type, which is in some sense ``better''. This is a bit silly if we already have the type we need as input, but we can make this idea work by introducing hints, which are types with holes, and insisting that the input is not a type, but merely a hint.

\end{frame}

\begin{frame}{Terms}

Note: red color marks differences from Bidirectional STLC.

\vspace{2em}

Terms: \\
$e ::=$ \\
\qquad $x \pipe \termdiff{\annot{e}{H}} \pipe $ \\
\qquad $\fun{x}{e} \pipe \app{e_1}{e_2} \pipe$ \\
\qquad $\pair{e_1}{e_2} \pipe \outl[e] \pipe \outr[e] \pipe$ \\
\qquad $\inl[e] \pipe \inr[e] \pipe \case{e}{e_1}{e_2} \pipe$ \\
\qquad $\unit \pipe \exfalso{e}$

\vspace{2em}

Typing contexts assign types to variables, but annotations in terms are hints, not necessarily types.

\end{frame}

\begin{frame}{Declarative typing -- differences}

\begin{center}
  $\infrule{\typing{e}{A} \quad \sidecond{\hintorder{H}{A}}}{\typing{\annot{e}{H}}{A}}$
\end{center}

\end{frame}

\begin{frame}{Hinting -- basic rules}

\begin{center}
  $\infrule[Var]{\sidecond{(x : A) \in \Gamma} \quad \sidecond{\hintorder{H}{A}}}{\hinting{x}{H}{A}}$

  \vspace{2em}

  $\infrule[Annot]{\hinting{e}{\combinehints{H_1}{H_2}}{A}}{\hinting{\annot{e}{H_1}}{H_2}{A}}$
\end{center}

\end{frame}

\begin{frame}{Hinting -- type-directed rules 1}

\begin{center}
  $\infrule{\sidecond{\combinehints{H}{(\Fun{\Hint}{\Hint})} = \Fun{A}{H'}} \quad \fullhinting{\extend{\Gamma}{x}{A}}{e}{H'}{B}}{\hinting{\fun{x}{e}}{H}{\Fun{A}{B}}}$

  \vspace{1em}

  $\infrule{\hinting{f}{\Fun{\Hint}{H}}{\Fun{A}{B}} \quad \hinting{a}{A}{A}}{\hinting{\app{f}{a}}{H}{B}}$

  \vspace{1em}

  $\infrule{\sidecond{\combinehints{H}{(\Prod{\Hint}{\Hint})} = \Prod{H_1}{H_2}} \quad \hinting{a}{H_1}{A} \quad \hinting{b}{H_2}{B}}{\hinting{\pair{a}{b}}{H}{\Prod{A}{B}}}$

  \vspace{1em}

  $\infrule{\hinting{e}{\Prod{H}{\Hint}}{\Prod{A}{B}}}{\hinting{\outl[e]}{H}{A}}$ \quad
  $\infrule{\hinting{e}{\Prod{H}{\Hint}}{\Prod{A}{B}}}{\hinting{\outr[e]}{H}{B}}$
\end{center}

\end{frame}

\begin{frame}{Hinting -- type-directed rules 2}

\begin{center}
  $\infrule{\sidecond{\combinehints{H}{(\Sum{\Hint}{\Hint})} = \Sum{H'}{B}} \quad \hinting{e}{H'}{A}}{\hinting{\inl[e]}{H}{\Sum{A}{B}}}$

  \vspace{2em}

  $\infrule{\sidecond{\combinehints{H}{(\Sum{\Hint}{\Hint})} = \Sum{A}{H'}} \quad \hinting{e}{H'}{B}}{\hinting{\inr[e]}{H}{\Sum{A}{B}}}$

  \vspace{2em}

  $\infrule{\hinting{e}{\Sum{\Hint}{\Hint}}{\Sum{A}{B}} \quad \begin{array}{c} \hinting{f}{\Fun{A}{H}}{\Fun{A}{C}} \\ \hinting{g}{\Fun{B}{C}}{\Fun{B}{C}} \end{array}}{\hinting{\case{e}{f}{g}}{H}{C}}$

  \vspace{2em}

  $\infrule{\sidecond{\hintorder{H}{\Unit}}}{\hinting{\unit}{H}{\Unit}}$ \quad
  $\infrule{\hinting{e}{\Empty}{\Empty}}{\hinting{\exfalso[e]}{A}{A}}$
\end{center}

\end{frame}

\begin{frame}{Metatheory}

Similarly to Extrinsic STLC, STLC with hints v1 does not enjoy uniqueness of typing. This is because we still can have terms like $\fun{x}{x}$ with hint $\Hint$, which can be typed with any type of the form $\Fun{A}{A}$. However, if the hint is informative enough, then the type is unique. Moreover, every typable term can be given a hint which makes its type unique.

\end{frame}

\section{Hinting v2}

\newcommand{\hintfor}[2]{#1\ \triangle\ #2}

\begin{frame}{Hinting v2}

STLC with hints (version 2) is a slight variation on the first one. The problem is that the rules of v1 look ugly, because we need to do so much management of the hints. To deal with this, we relegate all this work into a single separate rule (which uses a new operation), making all the remaining ones much more readable. The terms are the same as in STLC with hints v1.

\end{frame}

\begin{frame}{Hinting v2 -- hints for term constructors}

\begin{center}
  $\hintfor{H}{\fun{x}{e}} = \combinehints{H}{(\Fun{\Hint}{\Hint})}$ \\
  $\hintfor{H}{\pair{e_1}{e_2}} = \combinehints{H}{(\Prod{\Hint}{\Hint})}$ \\
  $\hintfor{H}{\inl[e]} = \combinehints{H}{(\Sum{\Hint}{\Hint})}$ \\
  $\hintfor{H}{\inr[e]} = \combinehints{H}{(\Sum{\Hint}{\Hint})}$ \\
  $\hintfor{H}{\unit} = \combinehints{H}{\Unit}$
\end{center}

\end{frame}

\begin{frame}{Hinting v2 -- basic rules}

\begin{center}
  $\infrule[Var]{\sidecond{(x : A) \in \Gamma} \quad \sidecond{\hintorder{H}{A}}}{\hinting{x}{H}{A}}$

  \vspace{2em}

  $\infrule[Annot]{\hinting{e}{\combinehints{H_1}{H_2}}{A}}{\hinting{\annot{e}{H_1}}{H_2}{A}}$

  \vspace{2em}

  $\infrule[HintFor]{\sidecond{e\ \texttt{constructor}} \quad \sidecond{\hintfor{H}{e} = H'} \quad \sidecond{H \neq H'} \quad \hinting{e}{H'}{A}}{\hinting{e}{H}{A}}$
\end{center}

\vspace{2em}

Note that because of the condition $H \neq H'$, the rule $\rulename{HintFor}$ can only be applied once, and after it is applied only the type-directed rules are available.

\end{frame}

\begin{frame}{Hinting v2 -- type-directed rules}
  
\begin{center}
  $\infrule{\fullhinting{\extend{\Gamma}{x}{A}}{e}{H}{B}}{\hinting{\fun{x}{e}}{\Fun{A}{H}}{\Fun{A}{B}}}$

  \vspace{1em}

  $\infrule{\hinting{f}{\Fun{\Hint}{H}}{\Fun{A}{B}} \quad \hinting{a}{A}{A}}{\hinting{\app{f}{a}}{H}{B}}$

  \vspace{1em}

  $\infrule{\hinting{a}{H_1}{A} \quad \hinting{b}{H_2}{B}}{\hinting{\pair{a}{b}}{\Prod{H_1}{H_2}}{\Prod{A}{B}}}$

  \vspace{1em}

  $\infrule{\hinting{e}{\Prod{H}{\Hint}}{\Prod{A}{B}}}{\hinting{\outl[e]}{H}{A}}$ \quad
  $\infrule{\hinting{e}{\Prod{H}{\Hint}}{\Prod{A}{B}}}{\hinting{\outr[e]}{H}{B}}$
\end{center}
  
\end{frame}
  
\begin{frame}{Hinting v2 -- type-directed rules}

\begin{center}
  $\infrule{\hinting{e}{H}{A}}{\hinting{\inl[e]}{\Sum{H}{B}}{\Sum{A}{B}}}$

  \vspace{2em}

  $\infrule{\hinting{e}{H}{B}}{\hinting{\inr[e]}{\Sum{A}{H}}{\Sum{A}{B}}}$

  \vspace{2em}

  $\infrule{\hinting{e}{\Sum{\Hint}{\Hint}}{\Sum{A}{B}} \quad \begin{array}{c} \hinting{f}{\Fun{A}{H}}{\Fun{A}{C}} \\ \hinting{g}{\Fun{B}{C}}{\Fun{B}{C}} \end{array}}{\hinting{\case{e}{f}{g}}{H}{C}}$

  \vspace{2em}

  $\infrule{}{\hinting{\unit}{\Unit}{\Unit}}$ \quad
  $\infrule{\hinting{e}{\Empty}{\Empty}}{\hinting{\exfalso[e]}{A}{A}}$
\end{center}

\end{frame}

\begin{frame}{Hinting v2 -- alternative rules}

\begin{center}
  $\infrule[AltApp]{\hinting{a}{?}{A} \quad \hinting{f}{\Fun{A}{H}}{\Fun{A}{B}}}{\hinting{\app{f}{a}}{H}{B}}$

  \vspace{2em}

  $\infrule[AltCase]{\begin{array}{c} \hinting{f}{\Fun{?}{H}}{\Fun{A}{C}} \\ \hinting{g}{\Fun{?}{C}}{\Fun{B}{C}} \end{array} \quad \hinting{e}{\Sum{A}{B}}{\Sum{A}{B}}}{\hinting{\case{e}{f}{g}}{H}{C}}$
\end{center}

\vspace{2em}

We could have made some different choices. For application, we could try to infer the argument type first and then feed it to the function as a hint. For case, we could try to infer domains of the branches first, then feed these as hints when checking the discriminee.

\end{frame}

\begin{frame}{Notations and derived terms}

We can introduce some handy notations:

\begin{itemize}
  \item $\check{e}{A} :\equiv \hinting{e}{A}{A}$
  \item $\infer{e}{A} :\equiv \hinting{e}{\Hint}{A}$
\end{itemize}

We can embed Intrinsic STLC terms:

\begin{itemize}
  \item $\ifun{x}{A}{e} :\equiv \annot{\fun{x}{e}}{\Fun{A}{\Hint}}$ \\
  \item $\iinl{B}{e} :\equiv \annot{\inl{e}}{\Sum{\Hint}{B}}$ \\
  \item $\iinr{A}{e} :\equiv \annot{\inr{e}}{\Sum{A}{\Hint}}$ \\
  \item $\iexfalso{A}{e} :\equiv \annot{\exfalso{e}}{A}$
\end{itemize}

We can also embed Dual Intrinsic STLC terms:

\begin{itemize}
  \item $\iapp{A}{f}{a} :\equiv \app{\annot{f}{\Fun{A}{\Hint}}}{a}$
  \item $\ioutl{B}{e} :\equiv \outl{\annot{e}{\Prod{\Hint}{B}}}$
  \item $\ioutr{A}{e} :\equiv \outl{\annot{e}{\Prod{A}{\Hint}}}$
  \item $\icase{A}{B}{e}{f}{g} :\equiv \case{\annot{e}{\Sum{A}{B}}}{f}{g}$
\end{itemize}

\end{frame}

\begin{frame}{Rules for derived terms}

\begin{center}
  $\infrule{\fullinfer{\extend{\Gamma}{x}{A}}{e}{B}}{\infer{\ifun{x}{A}{e}}{\Fun{A}{B}}}$ \quad
  $\infrule{\check{f}{\Fun{A}{B}} \quad \check{a}{A}}{\check{\iapp{A}{f}{a}}{B}}$

  \vspace{2em}

  $\infrule{\check{e}{\Prod{A}{B}}}{\check{\ioutl{B}{e}}{A}}$ \enspace
  $\infrule{\check{e}{\Prod{A}{B}}}{\check{\ioutr{A}{e}}{B}}$

  \vspace{2em}

  $\infrule{\infer{e}{A}}{\infer{\iinl{B}{e}}{\Sum{A}{B}}}$ \quad
  $\infrule{\infer{e}{B}}{\infer{\iinr{A}{e}}{\Sum{A}{B}}}$

  \vspace{2em}

  $\infrule{\check{e}{\Sum{A}{B}} \quad \check{f}{\Fun{A}{C}} \quad \check{g}{\Fun{B}{C}}}{\check{\icase{A}{B}{e}{f}{g}}{C}}$

  \vspace{2em}

  $\infrule{\infer{e}{\Empty}}{\infer{\iexfalso{A}{e}}{A}}$
\end{center}

\end{frame}

\begin{frame}{Metatheory}

Metatheoretically, STLC with hints v2 is similar to v1.

\end{frame}

\end{document}